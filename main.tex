\documentclass{article}
\usepackage[utf8]{vietnam}
\usepackage{amsmath}
\usepackage{systeme,mathtools}

\title{CHƯƠNG 4: TÍCH VÔ HƯỚNG, ĐỘ LỚN VÀ ĐỘ GIỐNG NHAU/ KHOẢNG CÁCH GIỮA CÁC VECTOR}
\author{Karlie Le }
\date{March 2019}

\begin{document}

\maketitle

\section{Hệ cơ sở => Không gian tọa độ: }

\begin{itemize}
    \item Ví dụ 1:
$R^{3}$ $\ni$ v = (x,y,z): Basis\large{u} = ($u_{x}$, $u_{y}$, $u_{z}$) = (i,j,k)

$u_{x}$ = (1,0,0)

$u_{y}$ = (0,1,0)

$u_{z}$ = (0,0,1)

=> $[v]_{u}$ = (x,y,z) 

 \textbf{Tổng quát:}
$R^{n}$ $\ni$ v = ($x_{1}$,..., $x_{n}$):

$u_{1}$ = (1,0,...,0) 

...

$u_{n}$ = (0,...,0,1)

=> $[v]_{u}$ = ($x_{1}$,..., $x_{n}$) 

    \item Ví dụ 2: $R^{2x2}$ $\ni$ v = 
    $\begin{bmatrix}
     a & b\\ 
     c & d
    \end{bmatrix}$
    
    $u_{1}$ = 
    $\begin{bmatrix}
     1 & 0\\ 
     0 & 0
    \end{bmatrix}$
;
    $u_{2}$ = 
    $\begin{bmatrix}
     0 & 1\\ 
     0 & 0
    \end{bmatrix}$
    
    $u_{3}$ = 
    $\begin{bmatrix}
     0 & 0\\ 
     1 & 0
    \end{bmatrix}$
;    
     $u_{4}$ = 
    $\begin{bmatrix}
     1 & 0\\ 
     0 & 1
    \end{bmatrix}$

=> $[v]_{u}$ = (a,b,c,d) $\in$ $R^{4}$

    \item Ví dụ 3: (Không gian đa thức)
    
$P_{n}$(R) $\ni$ v(z) = $\sum_{i=0}^{n}$ $a_{i}$$z^{i}$, $a_i$$\in$ R $\forall i$ = 0,...,n:

$\varepsilon$ = $ \{$$e_{i}$ = $z^{i}$\}$ _{i=0}^{n}$ 
=> $[v]_{\varepsilon}$ = ($a_{0}$,...$a_{n}$) $\in$ $R^{n+1}$

\underline{LƯU Ý:}
\begin{itemize}
    \item Mọi vector space đều có thể quy về $R^{n}$
    \item Dimension n là số basis vector nhỏ nhất để miêu tả $\forall$ v $\in$ V
\end{itemize}
\end{itemize}
\section{Hàm tuyến tính => Ma trận: }
\begin{itemize}
    \item Hàm f: V -> W (biến đổi từ không gian vector V qua W)
    
    Chọn Basis B cho V $\xrightarrow[space]{coordinate}$ $[v]_{B}$ $\in$ $R^{n}$, $\forall$v $\in$ V
    
    Chọn Basis D cho W $\xrightarrow[space]{coordinate}$ $[w]_{D}$ $\in$ $R^{m}$, $\forall$w $\in$ W
    
    \item f : V -> W là hàm tuyến tính (linear) nếu f(au + bv) = af(u) + bf (v)
    
    \textbf{"Cộng và scale đầu vào -> đầu ra cộng và scale tương ứng"}
    \item f(au + bv) $\neq$ af(u) + bf(v): hàm phi tuyến (non-linear)
    
    \underline{Ví dụ:} Hàm sigmoid $\sigma$(z)= $\frac{1}{1+e^{-z}}$ = $\frac{e^{z}}{e^{z}+1}$
    với z $\in$ $R^{n}$

\end{itemize}

\section{Một số phép toán ma trận: }

\begin{itemize}
    \item Ma trận A = [ $a_{ij}$ ] $\in$ $R^{mxn}$  gồm m hàng và n cột
    
    $A^{T}$ = [ $a_{ij}$ ]  $\in$ $R^{mxn}$. $(A+B)^T$ = $A^T$ + $B^T$
    
    \item Vector b theo quy ước mặc định là vector \textbf{cột}
    \item A = [$c_1$]...[$c_n$] gồm n cột, cột thứ j là $c_j$
    
    A = [$r_1^T$]...[$r_n^T$] gồm n cột, cột thứ j là $r_j^T$
    \item Các cột còn lại là độc lập tuyến tính (linear independent) khi không có cột nào là linear combination của các cột còn lại:
    
    $\sum_{j=1}^n$ $a_j$$c_j$ = $0_n$ $\Longleftrightarrow$ $a_j$=0 $\forall$j=1,...,n
    
    \item Matrix product: 
    AB = Q: [m,n][n,p] = [m.p]; $q_{ik}$ = $\sum_{j=1}^n$ $a_{ij}$$b_{jk}$
    
    Các vector cột: $q_k$ = $A_{b_k}$. $(AB)^T$ = $A^T$ $B^T$
    \item Tích vô hướng: u.v := $u^T$v = $\sum_i=1^m$ $u_i$$v_i$ $\in$ R \textbf{(inner/dot/scalar product)} 
    
    uxv:= ${uv}^T$ $\in$ $R^{mxm}$ \textbf{(Outer product)}
    \item Nghịch đảo $M^{-1}$ của ma trận vuông: M$M^{-1}$ = $l_m$ = diag(1,...,1). ${(AB)}^{-1}$ = $B^{-1}$$A^{-1}$
    
    \item Ma trận vuông đối xứng (symmetric): $H^T$ = H
    \item Ma trận vuông đối xứng M $\in$$R^{mxn}$ là xác định dương \textbf{(positive definite)} nếu $x^T$A>0, $\forall$ x$\neq$$0_m$.
    \item 
    M có trace tr(M):= $\sum_{i=1}^n$$a_{ij}$. 
    Frobenius norm $A_F$:=$\sqrt{\sum_{i=1}^m \sum_{j=1}^n a_y^2}$ 
    := $\sqrt{tr(A^T A)}$
    
    \item Quadratic function: 
    
    f(x) = a + $b^T$x + $\frac{1}{2}$$x_T$Cx = $\frac{1}{2}$$\sum_{i j=1}^n$ $c_ij x_i x_j$ + $\sum_{k=1}^n$ $b_k x_k$ + a
    
    với a $\in$ R, x,b $\in$ $R^n$. C $\in$ $R^{nxn}$ symmetric và ít nhất một hệ số $c_y \neq 0$
    
    \item Quadratic form:
    
    q(x)= $x^T Cx$ = $\sum_{i k=1}^n c_ij x_i x_j$ hàm của các đa thức cùng bậc 2 $x_i x_j$ 
    
\end{itemize}
\section{Tích vô hướng và khoảng cách của 2 vector trong $R^n$}
\begin{itemize}
    \item $a_i$ là mức độ giống/khác giữa v và basis vector $e_i$ trong V.
    \item a.b= $\sum_{i=1}^n a_i b_i$ $\in R$ là \textbf{mức độ liên quan} giữa a,b $\in R^n$
    
    => Là cơ sở để tính độ lớn, khoảng cách, góc,..., giữa các vectors.
\end{itemize}
\begin{itemize}
    \item Orthogonal (perpendicular) vectors $a \perp b$ => cos$\theta$ = 0 $\leftrightarrow$ a.b = 0
    \item Normalized (unit/direction) vector: ||b|| = 1
    \item Orthonormal vectors: a.b=0 và ||a|| = ||b|| 1
    
\end{itemize}
\section{Khái quát hóa lên abstract vector spaces}
\begin{itemize}
    \item Inner product tổng quát là 1 hàm <.,.>: V xV -> R thỏa 3 tính chất trong dot product
    \item Trong $R^n \ni$ a,b  và symmetric, positive-definite $\sum \in R^{nxn}$ :
    
    $<a,b>_{\sum}$ = $a^T \sum b$
\end{itemize}

\end{document}
