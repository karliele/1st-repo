\documentclass{article}
\usepackage[utf8]{vietnam}

\title{CHƯƠNG 1: GIỚI THIỆU VỀ TRÍ TUỆ NHÂN TẠO}
\author{lekieuanh30902 }
\date{March 2019}

\begin{document}

\maketitle 

\section{LƯỢC SỬ TRÍ TUỆ LOÀI NGƯỜI}
\begin{itemize}
    \item Lược sử trí tuệ loài người là lịch sử phát minh và sử dụng các công cụ
    \begin{itemize}
        \item Giúp giải phóng năng lượng và sức lao động (lửa, điện...)
        \item Giúp giải phóng thông tin kiến thức và kĩ năng (ngôn ngữ nói, ngôn ngữ viết, máy tính, Internet...)
        \item Giúp giải phóng trí tuệ (toán học, trí tuệ nhân tạo...)
    \end{itemize}
    \item Công nghệ phát triển vượt bậc
    
    Bùng nổ phát triển về thiết bị (các thiết bị nhỏ hơn, tiện hơn, rẻ hơn nhưng có nhiều công dụng hơn)

-> Liên kết với nhau qua Internet và thu thập dữ liệu gắn liền với đời sống

-> Tạo ra nguồn dữ liệu khổng lồ \textbf{Big Data} 
    \item  Trí tuệ nhân tạo (AI) được áp dụng trong gần như mọi lĩnh vực
    \begin{itemize}
        \item Áp dụng trong Health Care (Chăm sóc sức khỏe), Business (Kinh doanh), Advertisement (Quảng cáo), Cyber Security (An ninh mạng) ...
        \item Ứng dụng quan trọng nhất: \textbf{Tăng cường trí tuệ của con người} (Intelligence Augmentation)
    \end{itemize}
    \item Cuộc cách mạng AI đang đến?
    
    AI có thể sẽ thay thế chúng ta nghiên cứu làm việc trong mọi lĩnh vực nhờ khả năng giải phóng năng lượng, sức lao động, thông tin kiến thức, kĩ năng và trí tuệ. Khả năng vượt trội của AI cũng có thể là mối đe dọa cho nhân loại

=> "AI sẽ là phát minh quan trọng cuối cùng của loài người?"
\end{itemize}
\section{TRÍ TUỆ NHÂN TẠO}
\subsection{Trí tuệ (Intelligence) là gì?}

Là khả năng học và áp dụng kiến thức, kĩ năng.

Đối với ngành Trí tuệ nhân tạo và Học máy, trí tuệ bao gồm:

\begin{itemize}
    \item Causal reasoning (Khả năng suy luận từ kết quả)
    \item Planning (Khả năng lên kế hoạch)
    \item Creativity (Khả năng sáng tạo)
    \item Intuition (Khả năng sử dụng trực giác)
    \item Imagination (Khả năng tưởng tượng)
    \item Commonsense (Hiểu biết những kiến thức phổ cập)
\end{itemize}

\subsection{Trí tuệ nhân tạo (Aritificial Intelligence) là gì?}

AI là máy tính có một số chức năng về trí tuệ giống của con người

AI == Máy tính có khả năng học và áp dụng kiến thức, kĩ năng

Kiến thức, kĩ năng -> Hàm số -> Chương trình máy tính

\subsection{Học (Learning) là gì? Học máy (Machine learning) là gì?}
\begin{itemize}
    \item Học là khả năng thu thập kiến thức, kĩ năng thông qua trải nghiệm, giáo dục, nghiên cứu và tìm ra một hàm ẩn tối ưu với đầu vào (input) là lượng thông tin, kĩ năng đã thu thập, đầu ra (output) là kiến thức của riêng mình
    \item Học máy (Machine learning) là máy tính tự động học qua các trải nghiệm bằng cách tìm kiếm trong không gian hàm số/ chương trình, giúp máy tính tự lập trình qua các trải nghiệm.
\end{itemize}    

\textbf{Ví dụ về Machine Learning: AI phục hồi màu ảnh từ ảnh trắng đen:}

\begin{itemize}
    \item Đầu vào (Input): Ảnh trắng đen
    \item Đầu ra (Output): Ảnh màu
    \item Trải nghiệm: xem nhiều cặp ảnh trắng đen - ảnh màu -> học được các đặc điểm từ ảnh màu so với ảnh trắng đen
    \item Tạo một bộ dữ liệu (dataset) của các trải nghiệm gồm các cặp ảnh màu và ảnh trắng đen tương ứng -> Lập trình máy tính thực hiện phục hồi ảnh màu
    \item Cách đánh giá: Độ trùng khớp của ảnh chuẩn màu và ảnh phục hồi
\end{itemize}
\textbf{Khái quát: }Máy tính tìm trong bộ dữ liệu hàm số có ít sự khác biệt nhất giữa ảnh màu gốc và ảnh màu phục hồi -> tìm ra 1 hàm số tối ưu nhất

\textbf{Mô hình tổng quát của Machine Learning:}

Mô hình (TEFPA) - máy tính tự học qua các trải nghiệm:
\begin{itemize}
    \item T(Task): Cho nhiệm vụ
    \item E(Experience): Trải nghiệm
    \item F(Function): Không gian hàm số
    \item P(Performance): Chuẩn đánh giá
    \item A(Algorithm): Giải thuật
\end{itemize}

\textbf{Cho nhiệm vụ, dựa vào trải nghiệm, chuẩn đánh giá và giải thuật, máy tính tìm ra 1 hàm số trong không gian hàm số có độ khái quát cao nhất}

-> Máy tính học bằng cách tìm kiếm trong không gian hàm số/chương trình

Machine Learning giúp máy tính tự lập trình qua các trải nghiệm

\subsection{Hàm số:}
Để tìm 1 hàm số tối ưu có 2 vấn đề cần cân nhắc:
\begin{itemize}
    \item Vấn đề biểu diễn không gian hàm (Representation)
    \item Vấn đề tìm, huấn luyện (Search/Train/Optimize)
\end{itemize}

\end{document}
