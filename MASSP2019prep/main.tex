\documentclass{article}
\usepackage[utf8]{vietnam}
\usepackage{amsmath}
\usepackage{systeme,mathtools}

\title{HÀM SỐ VÀ THAM SỐ}
\author{Karlie Le }
\date{March 2019}

\begin{document}

\maketitle

\section{Ví dụ về hàm số}
\begin{itemize}
    \item $f_1$(x) = 2x
    \item $f_2$(x) = 2x -1
    \item Dạng tổng quát: \textbf{affine function} f(x) = ax + b, $x\in R$
    \item Quadratic functions: f(x) = $ax^2$ + bx + c với $a\neq 0$ 
    \item $a,b,c \in R$ là tham số của hàm, gọi chung là $\theta$
\end{itemize}
\section{Biểu diễn hàm (graphing)}
\begin{itemize}
    \item Biểu diễn trên 2D: 
    \begin{itemize}
        \item \underline{Cách 1:} 
        
        {(x,y= f(x))} 
        
        Vẽ những \textbf{cặp số (x,y)} với đầu vào là x và đầu ra là y tương ứng. Tập hợp những điểm (x,y) tạo ra đồ thị hàm số f(x).
        \item \underline{Cách 2:} 
        $$\left \{(x, \theta) \mid f(x;\theta) = const   \right \}$$
        
        + Vẽ trong không gian đầu vào và cố định giá trị đầu ra \textbf{(level sets/contour lines)}: Coi tham số cũng là một đầu vào. Chọn tham số $\theta$, có hàm số rồi chọn đầu vào x. Ta có đầu ra là $f(x;\theta)$.
        
        + Cố định giá trị đầu ra $f(x;\theta)$ => tập hợp những điểm của giá trị đầ vào $(x, \theta)$ 
        
        \textbf{Ví dụ:} Hàm số $y=f(\theta;x) = ax + b$
        
        x = 1 = const
        Xét các giá trị thay đổi của a,b sao cho hàm số $f(\theta;x)$ = const với x = 1. Tập hợp các điểm (a,b) thỏa mãn điều kiện trên là đồ thị hàm số $f(\theta;x)$
        
        \item Thay đổi giá trị của các tham số a,b, ứng với mỗi giá trị của (a,b), ta có thể vẽ được 1 hàm số f(x) = ax + b tương ứng 
    
    \end{itemize}
    \item Biểu diễn nhiều hàm số khác nhau: 
    
    \begin{itemize}
        \item Thay đổi giá trị của a => hàm số thay đổi độ dốc (xoay hàm số quanh gốc tọa độ O)
        \item Thay đổi giá trị của b => hàm số tịnh tiến 
    \end{itemize}
     => Các tham số a,b đóng vai trò xoay và tịnh tiến hàm số
     
     => Tập hợp các hàm số như trên tạo thành \textbf{Không gian các hàm số}
\item Các hàm số khác: 

        + $\sigma (z)= \frac{1}{1+e^{-z}}  = \frac{e^{z}}{e^{z} +1}$: 
        hàm logistic sigmoid.
        
        +  f(z) = tanh(z) = $2\sigma(2z) - 1 = \frac{e^{z}-e^{-z}}{e^{z}+e^{-z}} =\frac{1-e^{-2z}}{1+e^{-2z}}$
   
Dạng tổng quát: f(x) = $\sigma(ax+b)$, f(x) = tanh(ax+b)

\underline{\textbf{KHÁI QUÁT:}}  

+ Ta có thể dùng tham số và biến số qua các phép tính để miêu tả hàm số

Kí hiệu: $f(x,\theta)$ hoặc $f_\theta(x)$ với x là biến số, $\theta$ là tham số
+ Tham số $\theta$ thay đổi tạo ra nhiều hàm số khác nhau tạo thành không gian hàm F = $\left \{f(x;\theta) : \theta \in \ominus \right \}$
\end{itemize}
\section{Không gian hàm (function space): representation and search}
Để tìm hàm tối ưu trong không gian hàm: 
\begin{itemize}
    \item Cần đo độ \textbf{hướng} và \textbf{độ lớn} của hàm, \textbf{khoảng cách} (mức độ giống nhau giữa các hàm), xác định hướng di chuyển nhanh nhất, etc.
    
    => Giúp tìm kiếm các phần tử trong không gian hàm
    \item Biểu diễn các đại lượng này qua: VECTOR VÀ KHÔNG GIAN VECTOR
\end{itemize}
\end{document}
