\documentclass{article}
\usepackage[utf8]{vietnam}
\usepackage{amsmath}

\title{CHƯƠNG 3: VECTOR VÀ KHÔNG GIAN VECTOR TRỪU TƯỢNG}
\author{Kieu Anh Le }
\date{March 2019}

\begin{document}

\maketitle

\section{Vector:}
\begin{itemize}
    \item Vector trong hình học: hình 1 mũi tên (có gốc và ngọn, hướng, độ lớn)
    \item Vector trong không gian vector: hình 1 điểm
\end{itemize}
   
\section{Không gian vector:} 
\textbf{Vector space = 4-tuple <V, R, +, x>}
\begin{itemize}
    \item V: vector
    \item R: tập số thực
    \item +: phép cộng các vector
    \item x: phép nhân các vector
\end{itemize}

\underline{Định nghĩa:}

Nếu tập V cùng 2 phép toán thỏa mãn 2 điều kiện

\begin{enumerate}
    \item Cộng 2 phần tử: u+v $\in V $, u,v $\in V$ (phép tịnh tiến)
    \item Nhân vector với 1 số: a.v $\in V $ , a $\in  R $, v $\in V $
\end{enumerate}

\textbf{Thì V là không gian vector, và v $\in V $ là một vector.}

\emph{Tổng quát: Nếu ta tịnh tiến hoặc co giãn các vector trong V thì chúng vẫn thuộc v}


\large{VÍ DỤ:}

Mô tả 1 người (chiều cao, cân nặng, tuổi...)

=> Dùng 1 không gian tọa độ \textbf{coordinate space}:

V = $R^{n}$ = R x R x...x R => \textbf{Cartesian (direct) product}

v = ($v_{1}$...$v_{n}$) := [$v_{i}$]; $v_{i}$ $\in R$; $\forall i$ $\in$
{\{1...n}\}

\underline{Giải thích:}
\begin{itemize}
    \item Mỗi con số trong $v_{1}$...$v_{n}$ là các số thực đại diện cho các thông tin (VD: $v_{1}$: chiều cao; $v_{2}$: cân nặng,...)
    \item Phép toán kết nối các con số từ $v_{1}$ đến $v_{n}$ để tạo nên một vector gọi là \textbf{Cartesian direct product} (Tích trực tiếp)
    \item [$v_{i}$] là tập hợp các vector từ $v_{1}$ đến $v_{n}$
\end{itemize}

\underline{Quan Trọng:} Không gian tọa độ $R^{n}$ được dùng để tính toán cho \textbf{tất cả} các không gian vector trừu tượng khác.

\section{Phép cộng và Phép nhân vô hướng}

\begin{itemize}
    \item Phép cộng 2 vectors: 
    \begin{itemize}
        \item $v_{i}$ = $u_{i}$ + $w_{i}$
        \item v = v + w $\in V$
    \end{itemize}
    
    \item Phép nhân vô hướng vector với 1 số:
    \begin{itemize}
        \item $v_{i}$ = $\alpha $ x $u_{i}$
        \item v = $\alpha$ x u $\in$ V
    \end{itemize}
\end{itemize}

\section{Một số không gian vector <V,R,+,x> tiêu biểu}
\begin{itemize}
    \item Không gian các tensors (mảng đa chiều): 
    \begin{itemize}
       \item Tensor bậc 0 (scalars): R.
       \item Tensor bậc 1 (vectors): $R^{m}$  $\ni$ v = [$v_{i}$]. \item Tensor bậc 2 (ma trận): $R^{mxn}$  $\ni$ v =[$v_{ij}$] .
       \item Tensor bậc 3 V = $R^{mxnxp}$  $\ni $v = [$v_{ijk}$] ...
       
với [$v_{ijk}$] $\in$ R, i $\in$ {\{1...m}\}, j $\in$ {\{1...n}\}, k $\in$ {\{1...p}\}
    \end{itemize}
    \item \underline{Ví dụ:} Biểu diễn một số không gian hàm số (function spaces)

$P_{n}$(R): không gian các hàm đa thức của $x \in R$ có bậc nhỏ hơn hoặc bằng n

    f(z) = $a_{0}$ + $a_{1}$z + $a_{2}$$z^{2}$ + ... + $a_{n}$$z^{n}$ = 
         $\sum_{i=0}^{n}$ 
         $a_{i}$$z^{i}$ $\in$ $P_{n}$(R)
         
     g(z) =  $b_{0}$ + $b_{1}$z + $b_{2}$$z^{2}$ + ... + $b_{n}$$z^{n}$ = 
         $\sum_{i=0}^{n}$ 
         $b_{i}$$z^{i}$ $\in$ $P_{n}$(R)
 

\end{itemize}

\underline{Định nghĩa:}

+ Cộng 2 vectors: h = f + g; h(z) = 
$\sum_{i=0}^{n}$ ($a_{i}$ + $b_{i}$)$z_{i}$

+ Nhân scalar: v = $\alpha$f; v(z) = 
$\sum_{i=0}^{n}$$\alpha$
$a_{i}$$z^{i}$ $\in$ $P_{n}$(R)

=> Chứng minh được $P_{n}$ là 1 không gian vector

\underline{Lưu ý:} {\{1,z,$z^{2}$, ... , $z^{n}$}\} cũng là các vector trong $P_{n}$(R)

{\{$e_{i}$} \} = {\{$z_{i}$} \}$_{i=0}^{n}$ 
được gọi là \textbf{hệ cơ sở(basis)} của $P_{n}$(R)

$\forall$ v $\in$ V =  $P_{n}$(R): 
v = $a_{0}$$e_{0}$ + ... + $a_{n}$$e^{n}$ = 
$\sum_{i=0}^{n}$ $a_{i}$$e_{i}$

v = \textbf{linear combination (kết hợp tuyến tính)} của các cơ sở

\section{Vector cơ sở và hệ cơ sở là trọng tâm của Machine Learning}

\begin{itemize}
    \item Ý nghĩa của hệ cơ sở (basis) trong không gian vector:
    \begin{itemize}
        \item hướng, trục (directions)
        \item cột mốc (landmarks)
        \item đặc trưng (features)
        \item từ vựng (words)
        \item chuẩn, mẫu (prototypes, patterns, templates... )
        \item regularities, abstractions
    \end{itemize}
    \item Chọn và sắp xếp một hệ cơ sở (ordered basis):
    $\varepsilon $ = ($e_{1}$...$e_{n}$)
   
    $\overset{\forall v\in V }{\rightarrow}$ decomposition 
    [v] = coordinates ($a_{1}$,...$a_{n}$) $\in$ R
    
    \item \textbf{Lưu ý: }$a_{i}$ $\approx $ mức độ giống/khác (e.g frequency) giữa v và $e_{i}$
    
\end{itemize}
\end{document}

  


